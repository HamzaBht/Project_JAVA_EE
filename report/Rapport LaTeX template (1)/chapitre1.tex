\chapter{Présentation de la plateforme Fligght}

%Intro\footnotemark\\
\begin{spacing}{1.2}
%note en bas de page

\section{Introduction}

\par Ce chapitre introductif a pour objet de décrire la plateforme et son processus principal de comparaison.
%\begin{figure}[!h]
%\begin{center}
%remplacer "width" par "height" pour régler la hauteur
%\includegraphics[width=15cm]{presentation/schema}
%\end{center}
%légende de l'image
%\caption{Schéma descriptif}
%\end{figure}


\section{Présentation de "Fligght"}
La platforme est basée sur une api pour pouvoir accéder aux informations nécessaires sur les vols et les afficher au utilisateur.
Cette affichage est basé sur les filtres que ce dernier a choisi pour le vol désiré .
\\ \par
D'un point de vue utilisateur , ce dernier saisie en accédant à la plateforme toutes les informations qu'il souhaite a propos de son vol, pour que par la suite , toutes les vols qui sont convenables avec le résultat de recherche et comparaison s'affichent. 
\\ \par Ensuite , parmi ces résultats ce dernier présente le meilleur offre qui soit convenable avec ses critères , pour pouvoir lui lier à l'offre détaillée qui pourra par la suite la réserver sur le site officiel de la compagnie . Fligght inclue tous les vols que l'utilisateur cherchera , selon les possibilités horaires et budgétaires qu'il pourra chercher, tout dépendamment des offres annoncées par les compagnies responsables .

\end{spacing}
\newpage \par
\section{Conclusion}A propos du cycle de l'application , on peut dire que la tache déclenchante c'est celle de la saisie des informations du vol , puis recherche et comparaison , pour avoir les résultats qui lient l'utilisateur aux sites officiels des compagnies pour la réservation . Notre plateforme est censée en production inclure plus qu'une compagnie aérienne , mais pour ce projet on se limitera à une seule .